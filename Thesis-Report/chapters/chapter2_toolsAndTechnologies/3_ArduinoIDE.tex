\section{Arduino IDE}

Además del uso del lenguaje de programación C++, se empleó el Arduino IDE como una herramienta tecnológica clave. Arduino IDE es un entorno de desarrollo integrado que se utiliza ampliamente en la programación de microcontroladores, especialmente en plataformas como el ESP32. 

Este IDE ofrece una interfaz gráfica intuitiva para escribir, compilar y cargar código en el microcontrolador, lo que facilita el desarrollo de aplicaciones embebidas.

Se ha elegido Arduino IDE dado que es una herramienta utilizada en clase y permite una programación rápida y sencilla del ESP32. Sin embargo, existen otras alternativas como PlatformIO o ESP-IDF, que ofrece una mayor flexibilidad y control sobre el proceso de compilación y carga de código.