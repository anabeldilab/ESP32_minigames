\section{Conclusions}

This section presents the most important conclusions of this work.

Firstly, it should be highlighted that all objectives have been met, although not everything went as expected.

This project has demonstrated the feasibility of implementing a classic game like Snake on an ESP32 microcontroller, highlighting the importance of modularity and efficiency in programming embedded systems.

\subsection{Future Lines of Work}

This work opens several avenues for future research and developments in the field of embedded systems programming.

\begin{itemize}
    \item \textbf{Exploration of Other Platforms and Technologies:} It would be interesting to replicate this study on different microcontroller platforms to compare performance and capabilities.
    \item \textbf{Integration of Advanced Features:} There is potential to expand the game with additional features, such as network connectivity or artificial intelligence, to further explore the capabilities of the ESP32.
\end{itemize}

\subsection{Final Reflections}

The development of the Snake game on the ESP32 has provided valuable experience in the programming of embedded systems, highlighting both the challenges and opportunities in this field. This project not only contributes to academic knowledge in the area of embedded systems but also offers practical perspectives applicable to a variety of real-world applications.
