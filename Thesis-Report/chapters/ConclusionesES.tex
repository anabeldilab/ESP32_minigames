\section{Conclusiones}

En este apartado se presentan las conclusiones m\'as importantes de este trabajo. 

En primer lugar, se ha de destacar que se han cumplido todos los objetivos aunque no todo haya salido como se esperaba.

Este proyecto ha demostrado la viabilidad de implementar un juego clásico como Snake en un microcontrolador ESP32, destacando la importancia de la modularidad y eficiencia en la programación de sistemas embebidos.

\subsection{Lineas de Trabajo Futuras}

Este trabajo abre varias vías para investigaciones futuras y desarrollos en el campo de la programación de sistemas embebidos.

\begin{itemize}
    \item \textbf{Exploración de Otras Plataformas y Tecnologías:} Sería interesante replicar este estudio en diferentes plataformas de microcontroladores para comparar rendimientos y capacidades.
    \item \textbf{Integración de Funcionalidades Avanzadas:} Existe un potencial para expandir el juego con características adicionales, como conectividad en red o inteligencia artificial, para explorar más a fondo las capacidades del ESP32.
\end{itemize}

\subsection{Reflexiones Finales}

El desarrollo del juego Snake en el ESP32 ha proporcionado una valiosa experiencia en la programación de sistemas embebidos, destacando tanto los desafíos como las oportunidades en este campo. Este proyecto no solo contribuye al conocimiento académico en el área de sistemas embebidos, sino que también ofrece perspectivas prácticas aplicables a una variedad de aplicaciones en el mundo real.