\newpage 
\vspace*{200px}
\thispagestyle{empty}

\begin{abstract_en}
{\em

The present final project for the course of High-Level Programming Languages for Industrial Applications investigates the capability of the ESP32 microcontroller to execute modular, scalable, and memory-efficient software. Initially, a version of the Snake game was developed using the standard C++ library, focusing on readability and ease of modification. However, this first version failed to function properly on the ESP32, encountering stability problems and excessive memory usage. This setback led to the development of a second version, specifically optimized for memory efficiency and adapted to the capabilities of the ESP32. This work not only reveals the practical limitations of the ESP32 for certain programming approaches but also illustrates the importance of adaptability and optimization in the development of software for embedded systems.
}
\bigskip

\begin{keywords}

ESP32, Snake, C++, Modular, Efficiency, Memory, Microcontroller, SoC, Embedded, Programming, Optimization, Adaptability
\end{keywords}

\end{abstract_en}