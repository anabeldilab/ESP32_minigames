\section{Contexto y justificación}

\subsection{Contexto}
El proyecto se enmarca en el contexto de la creciente demanda de sistemas embebidos eficientes y flexibles. El ESP32, un sistema en chip (SoC) altamente versátil, se ha convertido en una plataforma popular para una amplia gama de aplicaciones, desde IoT hasta proyectos de automatización doméstica. La capacidad de este microcontrolador para manejar tareas complejas con eficiencia energética y conectividad integrada lo convierte en un candidato ideal para explorar prácticas de programación avanzadas.


\subsection{Justificación}
La justificación del proyecto radica en la necesidad de comprender mejor cómo se pueden desarrollar y optimizar códigos modulares en plataformas como el ESP32. La modularidad en el software es crucial para la escalabilidad y el mantenimiento eficientes, especialmente en sistemas embebidos donde los recursos son limitados. Además, la implementación de un juego como Snake proporciona un marco práctico y tangible para comparar diferentes enfoques de programación, centrándose en aspectos como la legibilidad del código, la eficiencia de la memoria y la estabilidad del sistema.

Este estudio se alinea con las tendencias actuales en la programación de microcontroladores, donde la eficiencia y la modularidad son fundamentales para el éxito de aplicaciones cada vez más sofisticadas y conectadas. Al comparar dos enfoques distintos de programación en un entorno de microcontrolador real, este trabajo contribuye al conocimiento práctico que puede ser aplicado en una variedad de contextos de sistemas embebidos, desde el desarrollo de dispositivos IoT hasta aplicaciones industriales avanzadas.