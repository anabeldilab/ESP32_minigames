\newpage 
\thispagestyle{empty}

\begin{abstract}
{\em

El presente trabajo final de la asignatura de Lenguajes de Alto Nivel para Aplicaciones Industriales 
investiga la capacidad del microcontrolador ESP32 para ejecutar software modular, escalable y eficiente en el uso de memoria. 
Inicialmente, se desarrolló una versión del juego Snake utilizando la librería estándar de C++, enfocada en la legibilidad y 
facilidad de modificación. Sin embargo, esta primera versión no logró funcionar adecuadamente en el ESP32, enfrentando problemas 
de estabilidad y uso excesivo de memoria. Este revés condujo al desarrollo de una segunda versión, optimizada específicamente 
para la eficiencia de memoria y adaptada a las capacidades del ESP32. Este trabajo no solo revela las limitaciones prácticas 
del ESP32 para ciertos enfoques de programación, sino que también ilustra la importancia de la adaptabilidad y la optimización 
en el desarrollo de software para sistemas embebidos.
}
\bigskip

\begin{palabrasClave}

ESP32, Snake, C++, Modular, Eficiencia, Memoria, Microcontrolador, SoC, Embebido, Programación, Optimización, Adaptabilidad
\end{palabrasClave}

\end{abstract}