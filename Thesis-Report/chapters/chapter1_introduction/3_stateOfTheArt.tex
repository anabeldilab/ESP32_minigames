\section{Estado del arte}

\subsection{Evolucion de los microcontroladores}

Los microcontroladores han experimentado una notable evolución desde su concepción en las décadas de 1960 y 1970. Los primeros dispositivos, como el Intel 4004 lanzado en 1971, representaron un gran avance en términos de integración y coste, permitiendo la digitalización de una amplia gama de dispositivos y procesos \cite{wikipedia_microcontroller}. 

Con el tiempo, estos microcontroladores se han desarrollado significativamente, mejorando en capacidad de procesamiento, memoria y funcionalidades integradas. La evolución ha permitido incorporar mayores velocidades de reloj, más memoria RAM y ROM, y soporte para una gama más amplia de entradas y salidas digitales y analógicas. 

En las últimas décadas, los microcontroladores han alcanzado capacidades aún mayores con la introducción de los sistemas en chip (SoC), como el ESP32, que integran no solo el CPU, sino también conectividad WiFi/Bluetooth y otras funciones especializadas \cite{semiengineering_evolution_mcu}. 

Hoy en día, los microcontroladores se encuentran en una multitud de dispositivos, desde electrodomésticos hasta sistemas industriales complejos, y continúan evolucionando, impulsando la innovación en áreas como la Internet de las Cosas (IoT), la automatización y la robótica \cite{wikipedia_microcontroller, semiengineering_evolution_mcu}.

\subsection{Programación de Microcontroladores}

El campo de la programación de microcontroladores ha evolucionado considerablemente, ofreciendo una amplia gama de lenguajes y herramientas para satisfacer diversas necesidades. Los lenguajes de programación comunes en este ámbito incluyen C, C++, Python, y lenguajes de bajo nivel como ensamblador.

C es un lenguaje ampliamente utilizado en sistemas embebidos, con estimaciones de la industria que indican que alrededor del 80\% de estos sistemas utilizan C. C++ comparte muchas características con C, pero también agrega capacidades orientadas a objetos y una biblioteca estándar que puede ahorrar tiempo en la escritura de código. Sin embargo, ambos lenguajes pueden ser complejos y requieren una comprensión técnica detallada \cite{qtio_microcontrollers}.

Python, aunque no tradicionalmente asociado con sistemas embebidos, ha ganado popularidad en aplicaciones de microcontroladores. Es conocido por su simplicidad y facilidad de uso, y es especialmente útil en la automatización de pruebas y en la recolección y análisis de datos \cite{allaboutcircuits_microcontrollers}.

El lenguaje ensamblador se utiliza en ciertas situaciones para programar directamente en los registros del procesador, proporcionando acceso directo al CPU y al hardware. Esto permite a los desarrolladores escribir código altamente optimizado para aplicaciones específicas, aunque puede ser desafiante de aprender y usar \cite{wevolver_microcontrollers}.

Además de estos lenguajes, también existen plataformas de microcontroladores como STM32, PIC, y AVR, cada una con sus propias ventajas y desventajas en términos de poder de procesamiento, eficiencia energética, soporte periférico y facilidad de uso.

Para una programación efectiva de microcontroladores, se requieren herramientas como Entornos de Desarrollo Integrados (IDEs), depuradores y emuladores, y sistemas de control de versiones. IDEs como Arduino IDE, MPLAB X y STM32CubeIDE proporcionan interfaces unificadas para escribir, depurar y desplegar código. Los depuradores y emuladores ayudan a identificar y solucionar problemas en el código, y los sistemas de control de versiones como Git, SVN y Mercurial permiten gestionar y documentar el código de manera eficiente.

\subsection{Tendencias Futuras en Microcontroladores y Programación Embebida}

Las tendencias futuras en microcontroladores y programación embebida se están centrando en varias áreas clave, como la eficiencia energética, el factor de forma más pequeño, la integración de inteligencia artificial (AI) y aprendizaje automático (ML), la seguridad de la red, y la electrónica automotriz.

Los fabricantes de circuitos integrados (IC) están adoptando métodos para reducir el consumo de energía de los microcontroladores, como la disminución de la frecuencia del reloj, la puerta de reloj, el escalado dinámico y el control individual de periféricos. Esto es especialmente importante en la electrónica de consumo, donde los dispositivos más livianos y pequeños son una tendencia.

En la automatización industrial, los microcontroladores confiables están diseñados para precisión y se utilizan para controlar robots industriales, herramientas de máquinas y tareas de automatización. Una tendencia emergente es el uso de IA y ML para mejorar la eficiencia y productividad de los robots industriales. Se espera que la mayoría de los microcontroladores industriales cuenten con unidades de procesamiento neural integradas para incorporar IA y aprendizaje profundo en sistemas de control y aplicaciones de automatización \cite{engineersgarage_microcontrollers_2023}.

La seguridad de la red es otra preocupación creciente, ya que los microcontroladores a menudo están conectados a Internet y operan en anchos de banda de datos bajos. Las soluciones para asegurar la comunicación de datos sobre dispositivos IoT incluyen protocolos de comunicación serial y comunicación encriptada a través de TLS y SSL. Las medidas de seguridad basadas en hardware, los protocolos de encriptación y las interfaces seriales son cada vez más importantes en aplicaciones basadas en microcontroladores \cite{engineersgarage_microcontrollers_2023}.

En la industria automotriz, los microcontroladores se están utilizando cada vez más para manejar las operaciones de unidades de control eléctrico, sistemas de falla segura y sistemas tolerantes a fallos automotrices. Las características avanzadas y únicas en desarrollo que dependen de los microcontroladores incluyen sistemas avanzados de asistencia al conductor, sistemas de infoentretenimiento, control automático del clima, sensores de estacionamiento y conducción autónoma \cite{engineersgarage_microcontrollers_2023}.

En el ámbito del software, la escasez de chips ha aumentado la popularidad de plataformas como Zephyr, que reduce la dependencia del software en la arquitectura de hardware del procesador, facilitando la portabilidad del firmware. Además, el marco ".NET nanoframework" permite crear aplicaciones C# para plataformas embebidas, y Rust está ganando terreno en la industria embebida, especialmente en el desarrollo de partes del kernel de Linux \cite{solwit_future_embedded_systems}.

Por último, el proyecto Micropython está experimentando un rápido desarrollo, con soporte para nuevas plataformas y una comunidad activa que contribuye al proyecto \cite{solwit_future_embedded_systems}.

Estas tendencias apuntan hacia una integración cada vez mayor de la tecnología de microcontroladores en diversos sectores, con un enfoque en la eficiencia, la conectividad y la seguridad.